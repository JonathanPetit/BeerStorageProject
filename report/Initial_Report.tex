\documentclass[10pt,a4paper]{article}
\usepackage[utf8]{inputenc}
\usepackage[francais]{babel}
\usepackage[T1]{fontenc}
\usepackage{amsmath}
\usepackage{amsfonts}
\usepackage{amssymb}
\usepackage[left=2cm,right=2cm,top=2cm,bottom=2cm]{geometry}

\begin{document}
\section{Description du projet}
Le projet consiste en un inventaire de bouteille de bière. L'application permet à l'utilisateur de visualiser son stock en temps réel, de le modifier et d'effectuer des recherches dans celui-ci. 

\section{Paradigme de qualité et conventions de codage}
\textbf{\textit{DRY}} - \textit{Don't Repeat Yourself}\\

Notre vision du paradigme : bien penser le code source de l'application pour limiter la répétition de code. Les relations entre les différents objets et leur responsabilité est donc très importante. 


\section{Métriques}
Pour évaluer de façon objective la qualité de notre code nous avons sélectionné plusieurs métriques qui seront surveillé durant la programmation. 

\subsection{Densité de commentaires}
Densité des commentaires (DC) par rapport aux lignes de code\\
DC = SLOC / CLOC (comment line of code)\\ \\

\textit{Entre 20\% et 40\%} 

\subsection{Couverture de code}
Le pourcentage de code couvert par des tests unitaires doit tendre vers 100\% pour avoir un code le plus facile à faire évoluer. 



\end{document}